\documentclass[12pt,a4paper]{report}
\usepackage[top = 1.6cm, left = 2.7cm, right = 2.7cm ]{geometry}
\usepackage{setspace}
\usepackage[utf8]{inputenc}
\usepackage[french]{babel}
\usepackage{float}
\usepackage[T1]{fontenc}
\usepackage{amsmath}
\usepackage{amsthm}
\usepackage{amsfonts}
\usepackage{amssymb}
\usepackage{graphicx}
\usepackage{hyperref}
\usepackage{amsthm,thmtools}
\usepackage{cite}
%\usepackage{bbm}
\usepackage{caption}
\usepackage{xcolor}
\usepackage{framed}
\usepackage{rotating}
%\usepackage[noend]{algorithmic}
%\usepackage{algorithm}
%\usepackage{minted}
\usepackage{cprotect}
\usepackage{caption}
\usepackage{fancyvrb}
\usepackage{afterpage}
\usepackage{subcaption}
\usepackage{hyperref}
%\usepackage[noend]{algcompatible}
\colorlet{shadecolor}{lightgray!25}

%% ALGORITHME

%\floatname{algorithm}{Algorithme}
%\renewcommand{\algorithmicrequire}{\textbf{Entrée :}}
%\renewcommand{\algorithmicensure}{\textbf{Sortie :}}
%\renewcommand{\algorithmicif}{\textbf{si}}
%\renewcommand{\algorithmicthen}{\textbf{alors}}
%\renewcommand{\algorithmicelse}{\textbf{sinon}}
%\renewcommand{\algorithmicelsif}{\textbf{sinon si}}
%\renewcommand{\algorithmicwhile}{\textbf{tant que}}
%\renewcommand{\algorithmicdo}{\textbf{faire}}
%\renewcommand{\algorithmicend}{\textbf{fin}}
%\renewcommand{\algorithmicreturn}{\textbf{retourner}}
%\renewcommand{\algorithmicfor}{\textbf{pour}}
%\renewcommand{\algorithmicforall}{\textbf{pour tout}}


\renewcommand{\thefootnote}{\textit{\alph{footnote}}}
%% Theorem Styles %%

\theoremstyle{definition}%{3pt}{3pt}{\slshape}{}{\bfseries}{.}{.5em}{}
\newtheorem{definition}{Définition}[chapter]
\newtheorem{theorem}{Théorème}[chapter]
\newtheorem{lemma}{Lemme}[chapter]
\newtheorem*{notation}{Notation}
\newtheorem*{notations}{Notations}
\newtheorem{propriete}{Propriété}[chapter]
\newtheorem*{proprietes}{Propriétés}
\newtheorem{proposition}{Proposition}[chapter]
\newtheorem{corollaire}{Corollaire}[chapter]
\theoremstyle{remark}
\newtheorem{example}{Exemple}[chapter]
\newtheorem{remark}{Remarque}[chapter]

%%Definition by chapter%%

\usepackage{etoolbox}
\makeatletter
\patchcmd\thmtlo@chaptervspacehack
{\addtocontents{loe}{\protect\addvspace{10\p@}}}
{\addtocontents{loe}{\protect\thmlopatch@endchapter\protect\thmlopatch@chapter{\thechapter}}}
{}{}
\AtEndDocument{\addtocontents{loe}{\protect\thmlopatch@endchapter}}
\long\def\thmlopatch@chapter#1#2\thmlopatch@endchapter{%
	\setbox\z@=\vbox{#2}%
	\ifdim\ht\z@>\z@
	\hbox{\bfseries\chaptername\ #1}\nobreak
	#2
	\addvspace{10\p@}
	\fi
}
\def\thmlopatch@endchapter{}

\makeatother
\renewcommand{\thmtformatoptarg}[1]{ -- #1}

%% Commandes persos %%
\newcommand\bbone{\ensuremath{\mathbbm{1}}}
\newcommand{\eg}{e.g., }
\newcommand{\ssi}{ssi }
\newcommand{\ie}{i.e., }
\newcommand{\cf}{cf. }
\newcommand{\pr}{\mathbb{P}}
\renewcommand{\listtheoremname}{Définitions et théorèmes}

\setstretch{1,1}
\let\labelitemi\labelitemii


\begin{document}
\begin{titlepage}

	\newcommand{\HRule}{\rule{\linewidth}{0.5mm}} % Defines a new command for the horizontal lines, change thickness here

	\center % Center everything on the page
	%----------------------------------------------------------------------------------------
	%	HEADING SECTIONS
	%----------------------------------------------------------------------------------------


	\begin{center}
	\includegraphics[height=2cm]{logos/UMONS_FS.pdf}
	\hspace{5cm}
	\includegraphics[height=1.7cm]{logos/UMONS+txt}
	\\[1em]
	\vspace{1cm}
	\end{center}
	\textsc{\large Service de génie logiciel }\\[0.5cm] % Major heading such as course name

	%\vspace{1cm}
	%----------------------------------------------------------------------------------------
	%	TITLE SECTION
	%----------------------------------------------------------------------------------------

	\vspace{0.5cm}
	\HRule \\[0.4cm]
	{\huge \bfseries \centering \quad Empirical Analysis of Development Bots in Social Coding Repositories}\\[0.4cm] % Title of your document
	\HRule \\[1cm]
	%{\large Projet de Master 1}

	%----------------------------------------------------------------------------------------
	%	AUTHOR SECTION//
	%----------------------------------------------------------------------------------------
	\vspace{2cm}
	\begin{minipage}{0.4\textwidth}
		\begin{flushleft} \large
			\emph{Auteur :} \\Robin Willième
		\end{flushleft}
	\end{minipage}
	~
	\begin{minipage}{0.4\textwidth}
		\begin{flushright} \large
			\quad \\
			\emph{Directeurs:}\\ \quad Tom Mens \\ \quad Alexandre Decan\\	\vspace*{0.5cm}
			%\emph{Rapporteur:}\\
			%Quentin Hautem
		\end{flushright}

	\end{minipage}\\[5cm]

	% If you don't want a supervisor, uncomment the two lines below and remove the section above
	%\Large \emph{Author:}\\
	%John \textsc{Smith}\\[3cm] % Your name

	%----------------------------------------------------------------------------------------
	%	DATE SECTION
	%----------------------------------------------------------------------------------------
	\vspace{3.7cm}
	%{\large \today}\\[3cm] % Date, change the \today to a set date if you want to be precise
	\begin{center}
			Ann\'ee acad\'emique 2019-2020
	\end{center}

	%----------------------------------------------------------------------------------------
	%	LOGO SECTION
	%----------------------------------------------------------------------------------------
	%----------------------------------------------------------------------------------------

	\vfill % Fill the rest of the page with whitespace

\end{titlepage}

%\pagenumbering{gobble}
%\shipout\null
%\begin{flushright}
%{\Large \textbf{Remerciements}} \\
%\end{flushright}

\begin{flushright}
Je remercie toutes les personnes 
\end{flushright}
%\thispagestyle{empty}

\afterpage{\null\newpage}

\tableofcontents

\afterpage{\null\newpage}

\chapter{Introduction}
\pagenumbering{arabic}

\chapter{Etat de l'art}


\bibliographystyle{plain}
\bibliography{bibli}

\end{document}
